\documentclass[11pt,a4paper]{article}
\input{config}

\begin{document}
\titlepageandcontents

%---------------------------------------------------------------------------
\section{Zadání}

Zde napište informace k zadání (nejde jen o přepis toho, co je na webu;
komentujte vaše vlastní zpřesnění zadání, zaměření, důrazy, pojetí atd.). Text
strukturujte, použijte odrážky, číslování$\ldots$

Rozsah: cca 10 odrážek

%---------------------------------------------------------------------------
\section{Použité technologie}

\section{Technologie potřebné pro běh}
\begin{itemize}
	\item Grafický akcelerátor s podporou OpenGL 4.3
		\begin{itemize}
			\item SSBO
			\item Image load/store
			\item Compute shadery
		\end{itemize}
	\item SDL2
	\item GLM
\end{itemize}

\section{Technologie použité pro tvorbu}
\begin{itemize}
	\item Blender
	\item Textový editor/vývojové prostředí vlastní volby
\end{itemize}

%---------------------------------------------------------------------------
\section{Použité zdroje}

Zde vypište, které zdroje jste použili k tvorbě: hotový kód, hotová data
(obrázky, modely, $\ldots$), studijní materiály. Pokud vyplyne, že v projektu
je použit kód nebo data, která nejsou uvedena tady, jedná se o závažný problém
a projekt bude pravděpodobně hodnocen 0 body.

Rozsah: potřebný počet odrážek

%---------------------------------------------------------------------------
\section{Nejdůležitější dosažené výsledky}

Popište 3 věci, které jsou na vašem projektu nejlepší. Nejlépe ukažte a
komentujte obrázky, v nejhorším případě vypište textově.

%---------------------------------------------------------------------------
\section{Ovládání vytvořeného programu}

\begin{itemize}
	\item Levé tlačítko myši - Společně s pohybem myši umožňuje změnu úhlu pohledu na scénu.
	\item Kolečko myši - Přiblížení a oddálení zobrazené scény.
	\item Klávesa R - Přepíná autonomní otáčení zobrazené scény.
	\item Klávesa T - Přepíná zobrazení vypočteného stínového tělesa.
\end{itemize}

%---------------------------------------------------------------------------
\section{Zvláštní použité znalosti}

Použití OpenGL compute shaderů se různě liší od použití OpenCL či CUDA. To vyžadovalo získání dodatečných informací o jejich využívání a o komunikaci s němi.

%---------------------------------------------------------------------------
\section{Rozdělení práce v týmu}

\paragraph{Zdeněk Biberle} Základ aplikace, compute shader pro výpočet stínového tělesa.
\paragraph{Vít Hodes} Vykreslení scény za použití stínového tělesa, nahrávání modelů.

%---------------------------------------------------------------------------
\section{Co bylo nejpracnější}

\paragraph{Zdeněk Biberle}

Značnou část času jsem strávil nad komunikací mezi aplikací a compute shaderů. Později jsem zjistil, že implementace SSBO na grafických kartách od AMD se v současnosti vyznačuje různými problémy a tudíž jsem řešil nevyřešitelné.

\paragraph{Vít Hodes} TODO Něco sem

%---------------------------------------------------------------------------
\section{Zkušenosti získané řešením projektu}

\paragraph{Zdeněk Biberle}
\begin{itemize}
	\item Compute shadery
	\item SSBO
\end{itemize}

\paragraph{Vít Hodes}
\begin{itemize}
	\item Něco sem
\end{itemize}

%---------------------------------------------------------------------------
\section{Autoevaluace}

Ohodnoťte vaše řešení v jednotlivých kategoriích (0 – nic neuděláno,
zoufalství, 100\% – dokonalost sama). Projekt, který ve finále obdrží plný
počet bodů, může mít složky hodnocené i hodně nízko. Uvedení hodnot blízkých
100\% ve všech nebo mnoha kategoriích může ukazovat na nepochopení problematiky
nebo na snahu kamuflovat slabé stránky projektu. Bodově hodnocena bude i
schopnost vnímat silné a slabé stránky svého řešení.

\paragraph{Technický návrh (50\%):} (analýza, dekompozice problému, volba
vhodných prostředků, $\ldots$) 
Stručně (1-2 řádky) komentujte hodnocení. 

\paragraph{Programování (50\%):} (kvalita a čitelnost kódu, spolehlivost běhu,
obecnost řešení, znovupoužitelnost, $\ldots$)
Stručně (1-2 řádky) komentujte hodnocení. 

\paragraph{Vzhled vytvořeného řešení (50\%):} (uvěřitelnost zobrazení,
estetická kvalita, vhled GUI, $\ldots$)
Stručně (1-2 řádky) komentujte hodnocení. 

\paragraph{Využití zdrojů (50\%):} (využití existujícího kódu a dat, využití
literatury, $\ldots$)
Stručně (1-2 řádky) komentujte hodnocení. 

\paragraph{Hospodaření s časem (50\%):} (rovnoměrné dotažení částí projektu,
míra spěchu, chybějící části řešení, $\ldots$)
Stručně (1-2 řádky) komentujte hodnocení. 

\paragraph{Spolupráce v týmu (50\%):} (komunikace, dodržování dohod, vzájemné
spolehnutí, rovnoměrnost, $\ldots$)
Stručně (1-2 řádky) komentujte hodnocení. 

\paragraph{Celkový dojem (50\%):} (pracnost, získané dovednosti, užitečnost,
volba zadání, cokoliv, $\ldots$)
Stručně (5-10 řádků) komentujte hodnocení. 

%---------------------------------------------------------------------------
\section{Doporučení pro budoucí zadávání projektů}

Témata projektů byla dostatečně pestrá, ale příště by mohla být zadána dříve.

%---------------------------------------------------------------------------
\section{Různé}

Ještě něco by v dokumentaci mělo být? Napište to sem! Podle potřeby i založte
novou kapitolu.

\end{document}
% vim:set ft=tex expandtab enc=utf8:
