\documentclass[11pt,a4paper]{article}
\input{config}

\begin{document}
\titlepageandcontents

%---------------------------------------------------------------------------
\section{Zadání}

Implementace generovaní shadow volumes siluet pomocí compute shaderů a následné využití těchto siluet
pro tvorbu stínů pomocí z-fail algoritmu.

%---------------------------------------------------------------------------
\section{Použité technologie}

\section{Technologie potřebné pro běh}
\begin{itemize}
	\item Grafický akcelerátor s podporou OpenGL 4.3
		\begin{itemize}
			\item SSBO
			\item Image load/store
			\item Compute shadery
		\end{itemize}
	\item SDL2
	\item GLM
	\item glew
\end{itemize}

\section{Technologie použité pro tvorbu}
\begin{itemize}
	\item Blender
	\item Textový editor/vývojové prostředí vlastní volby
\end{itemize}

%---------------------------------------------------------------------------
\section{Použité zdroje}
	\begin{itemize}
		\item Stanford bunny
	\end{itemize}
%---------------------------------------------------------------------------
\section{Nejdůležitější dosažené výsledky}

Funguje generování shadow volume na Nvidia kartách.

%---------------------------------------------------------------------------
\section{Ovládání vytvořeného programu}

\begin{itemize}
	\item Levé tlačítko myši - Společně s pohybem myši umožňuje změnu úhlu pohledu na scénu.
	\item Kolečko myši - Přiblížení a oddálení zobrazené scény.
	\item Klávesa R - Přepíná autonomní otáčení zobrazené scény.
	\item Klávesa T - Přepíná zobrazení vypočteného stínového tělesa.
\end{itemize}

%---------------------------------------------------------------------------
\section{Zvláštní použité znalosti}

Použití OpenGL compute shaderů se různě liší od použití OpenCL či CUDA. To vyžadovalo získání dodatečných informací o jejich využívání a o komunikaci s němi.
Využití image load/store v shaderech openGL pro vytvoření vlastního "stencil bufferu".

%---------------------------------------------------------------------------
\section{Rozdělení práce v týmu}

\paragraph{Zdeněk Biberle} Základ aplikace, compute shader pro výpočet stínového tělesa.
\paragraph{Vít Hodes} Vykreslení scény za použití stínového tělesa, nahrávání modelů.

%---------------------------------------------------------------------------
\section{Co bylo nejpracnější}

\paragraph{Zdeněk Biberle}

Značnou část času jsem strávil nad komunikací mezi aplikací a compute shaderů. Později jsem zjistil, že implementace SSBO na grafických kartách od AMD se v současnosti vyznačuje různými problémy a tudíž jsem řešil nevyřešitelné.

\paragraph{Vít Hodes} Přijít na to, jak impementovat celočíselný "stencil buffer" s obecným přístupem ze shaderů. Po prozkoumání několika slepých cest se dospělo k řešení pomocí iimage2D a atomického sčítání. Taky implementovat stíny nad shadow volume, který se mi na AMD kartě negeneruje správně se moc nedá.

%---------------------------------------------------------------------------
\section{Zkušenosti získané řešením projektu}

\paragraph{Zdeněk Biberle}
\begin{itemize}
	\item Compute shadery
	\item SSBO
\end{itemize}

\paragraph{Vít Hodes}
\begin{itemize}
	\item Framebuffer object, color/depth attachment
	\item v GLSL existují i shadow samplery a image samplery
	\item image load/store, atomické operace obecně v shaderech
	
\end{itemize}

%---------------------------------------------------------------------------
\section{Autoevaluace}

\paragraph{Technický návrh (50\%):} (analýza, dekompozice problému, volba
vhodných prostředků, $\ldots$) 
Stručně (1-2 řádky) komentujte hodnocení. 

\paragraph{Programování (50\%):} (kvalita a čitelnost kódu, spolehlivost běhu,
obecnost řešení, znovupoužitelnost, $\ldots$)
Stručně (1-2 řádky) komentujte hodnocení. 

\paragraph{Vzhled vytvořeného řešení (50\%):} (uvěřitelnost zobrazení,
estetická kvalita, vhled GUI, $\ldots$)
Stručně (1-2 řádky) komentujte hodnocení. 

\paragraph{Využití zdrojů (50\%):} (využití existujícího kódu a dat, využití
literatury, $\ldots$)
Stručně (1-2 řádky) komentujte hodnocení. 

\paragraph{Hospodaření s časem (50\%):} (rovnoměrné dotažení částí projektu,
míra spěchu, chybějící části řešení, $\ldots$)
Stručně (1-2 řádky) komentujte hodnocení. 

\paragraph{Spolupráce v týmu (50\%):} (komunikace, dodržování dohod, vzájemné
spolehnutí, rovnoměrnost, $\ldots$)
Stručně (1-2 řádky) komentujte hodnocení. 

\paragraph{Celkový dojem (50\%):} (pracnost, získané dovednosti, užitečnost,
volba zadání, cokoliv, $\ldots$)
Stručně (5-10 řádků) komentujte hodnocení. 

%---------------------------------------------------------------------------
\section{Doporučení pro budoucí zadávání projektů}

Témata projektů byla dostatečně pestrá, ale příště by mohla být zadána dříve.

%---------------------------------------------------------------------------
\section{Různé}

Ještě něco by v dokumentaci mělo být? Napište to sem! Podle potřeby i založte
novou kapitolu.

\end{document}
% vim:set ft=tex expandtab enc=utf8:
