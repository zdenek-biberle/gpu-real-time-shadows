\documentclass[11pt,a4paper]{article}
\input{config}

\begin{document}
\titlepageandcontents

%---------------------------------------------------------------------------
\section{Zadání}

\begin{itemize}
	\item Výpočet stínových těles
		\begin{itemize}
			\item Implementace na GPU
			\item Implementace na CPU pro porovnání rychlosti
		\end{itemize}
	\item Vytvoření demonstrační aplikace
		\begin{itemize}
			\item Vykreslení scény se stíny pomocí patřičně modifikovaného z-fail algoritmu
			\item Měření rychlosti
			\item Různé vizualizační pomůcky
		\end{itemize} vypočítaná stínová tělesa na jendoduché scéně
\end{itemize}

%---------------------------------------------------------------------------
\section{Použité technologie}

\section{Technologie potřebné pro běh}
\begin{itemize}
	\item Grafický akcelerátor s podporou OpenGL 4.3
		\begin{itemize}
			\item SSBO
			\item Image load/store
			\item Compute shadery
		\end{itemize}
	\item SDL2
	\item GLM
	\item glew
\end{itemize}

\section{Technologie použité pro tvorbu}
\begin{itemize}
	\item Blender
	\item Textový editor/vývojové prostředí vlastní volby
\end{itemize}

%---------------------------------------------------------------------------
\section{Použité zdroje}
\begin{itemize}
	\item Stanford bunny
	\item Utah teapot
	\item Byungmoon Kim, Kihwan Kim, Greg Turk - Real Time Shadow of Transparent Casters Using Shadow Volume, 2007
	\item Dokumentace OpenGL 4.3
\end{itemize}
%---------------------------------------------------------------------------
\section{Nejdůležitější dosažené výsledky}

\begin{figure}[h]
	\captionsetup{type=figure}
	\includegraphics[width=\textwidth]{images/bunny.png}
	\captionof{figure}{Králíček vrhající stín na terén}
\end{figure}

\begin{figure}[h]
	\captionsetup{type=figure}
	\includegraphics[width=\textwidth]{images/bunny-volume.png}
	\captionof{figure}{Králíček a jeho stínové těleso}
\end{figure}

Aplikace provádí generování stínového tělesa pro libovolný model na CPU i na GPU a scénu poté zobrazuje s jeho použitím. Dodatečně umožňuje i zobrazení samotného stínového tělesa. Ovšem dle našich zkušeností nefunguje generování stínového tělesa na grafických akrtách od společnosti AMD.

\subsection{Generování stínových těles i pro non-manifold geometrii}

\begin{figure}[h]
	\captionsetup{type=figure}
	\includegraphics[width=\textwidth]{images/multiple-raised-planes.png}
	\captionof{figure}{Geometrie složená z několika non-manifold částí vrhající stíny}
	\label{fig:multiple-raised-planes}
\end{figure}

Zadaný algoritmus umožňuje generování stínových těles i pro non-manifold geometrii. Na obrázku \ref{fig:multiple-raised-planes} vidíme několik čtverců vrhajících korektní stín. Každý tento čtverec je složen pouze ze dvou trojúhelníků a není tedy možné na ně aplikovat běžný algoritmus generování stínových těles (tj. protažení hran mezi dvěma trojúhelníky, kde jeden je natočen ke světlu a druhý je od světla odvrácen) tak, aby dosáhl správného výsledku.


%---------------------------------------------------------------------------
\section{Ovládání vytvořeného programu}

\begin{itemize}
	\item Levé tlačítko myši - Společně s pohybem myši umožňuje změnu úhlu pohledu na scénu.
	\item Kolečko myši - Přiblížení a oddálení zobrazené scény.
	\item Klávesa R - Přepíná autonomní otáčení zobrazené scény.
	\item Klávesa T - Přepíná zobrazení vypočteného stínového tělesa.
	\item Klávesa C - Přepíná mezi CPU a GPU implementací výpočtu stínového tělesa.
\end{itemize}

%---------------------------------------------------------------------------
\section{Zvláštní použité znalosti}

Použití OpenGL compute shaderů se různě liší od použití OpenCL či CUDA. To vyžadovalo získání dodatečných informací o jejich využívání a o komunikaci s němi.
Využití image load/store v shaderech openGL pro vytvoření vlastního "stencil bufferu".

%---------------------------------------------------------------------------
\section{Rozdělení práce v týmu}

\paragraph{Zdeněk Biberle} Základ aplikace, GPU výpočet stínového tělesa.
\paragraph{Vít Hodes} Vykreslení scény za použití stínového tělesa, nahrávání modelů, CPU výpočet stínového tělesa.

%---------------------------------------------------------------------------
\section{Co bylo nejpracnější}

\paragraph{Zdeněk Biberle}
Značnou část času jsem strávil nad komunikací mezi aplikací a compute shaderů. Později jsem zjistil, že implementace SSBO na grafických kartách od AMD se v současnosti vyznačuje různými problémy a tudíž jsem řešil nevyřešitelné.

\paragraph{Vít Hodes} Přijít na to, jak impementovat celočíselný "stencil buffer" s obecným přístupem ze shaderů. Po prozkoumání několika slepých cest se dospělo k řešení pomocí iimage2D a atomického sčítání. Taky implementovat stíny nad shadow volume, který se mi na AMD kartě negeneruje správně se moc nedá.

%---------------------------------------------------------------------------
\section{Zkušenosti získané řešením projektu}

\paragraph{Zdeněk Biberle}
\begin{itemize}
	\item Compute shadery
	\item SSBO
\end{itemize}

\paragraph{Vít Hodes}
\begin{itemize}
	\item Framebuffer object, color/depth attachment
	\item v GLSL existují i shadow samplery a image samplery
	\item image load/store, atomické operace obecně v shaderech
	
\end{itemize}

%---------------------------------------------------------------------------
\section{Autoevaluace}

\paragraph{Technický návrh (70\%):} (analýza, dekompozice problému, volba
vhodných prostředků, $\ldots$) 


\paragraph{Programování (70\%):} (kvalita a čitelnost kódu, spolehlivost běhu,
obecnost řešení, znovupoužitelnost, $\ldots$)
Vzhledem ke kratšímu rozsahu se moc neuplatnuje zapouzdření ve vykreslovací části. Znovupoužitelnost
spíše znalostí než konkrétního kódu.

\paragraph{Vzhled vytvořeného řešení (40\%):} (uvěřitelnost zobrazení,
estetická kvalita, vhled GUI, $\ldots$)
Estetická kvalita nebyla cílem řešení.

\paragraph{Využití zdrojů (70\%):} (využití existujícího kódu a dat, využití
literatury, $\ldots$)
Bez dokumentací OpenGL API by tento projekt nevznikl.

\paragraph{Hospodaření s časem (50\%):} (rovnoměrné dotažení částí projektu,
míra spěchu, chybějící části řešení, $\ldots$)
Přiměřené. 

\paragraph{Spolupráce v týmu (80\%):} (komunikace, dodržování dohod, vzájemné
spolehnutí, rovnoměrnost, $\ldots$)
Komunikace dobrá.

\paragraph{Celkový dojem (70\%):} (pracnost, získané dovednosti, užitečnost,
volba zadání, cokoliv, $\ldots$)
Projekt byl zajímavý a potenciálně velmi užitečný. Škoda problémů s různými platformami(AMD x nVidia).

%---------------------------------------------------------------------------
\section{Doporučení pro budoucí zadávání projektů}

Témata projektů byla dostatečně pestrá, ale příště by mohla být zadána dříve.

%---------------------------------------------------------------------------
\section{Různé}

Ještě něco by v dokumentaci mělo být? Napište to sem! Podle potřeby i založte
novou kapitolu.

\end{document}
% vim:set ft=tex expandtab enc=utf8:
